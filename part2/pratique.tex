\chap{La pratique}

\section{Le support}
Le support faisant partie intégrante du service informatique chez Kalicommunication, j'ai pu concrétiser ce que les développeurs de DataXplorer m'avaient transmis. De manière générale, le support constitue tous les problèmes urgents que rencontrent les clients ou les différents services de Kalicommunication. Les tâches de support demandent très peu de développements, voir dans certains cas, aucun.\newline
Pour pouvoir répondre aux demandes dans les plus brefs délais, nous avons utilisé un logiciel de gestion de tickets. Les tickets représentent les demandes des services de l'entreprise au service informatique, que ce soit pour une correction ou pour une évolution. Ils permettent, entre autres, de se substituer des mails et de pouvoir garder, facilement, une trace des demandes.\newline
À mes débuts, je consacrais 100\% de mon temps au support. En effet, résoudre un problème mineur pouvait me prendre plusieurs heures en raison de mon manque d'expérience. Par la pratique, les concepts qui m'étaient encore flous commençaient à s'éclaircir. Le temps consacré résoudre les problèmes de supports diminuait. J'ai donc pu me concentrer sur le développement le reste du temps.

\section{Le développement}
Comme évoqué dans la section présentant l'architecture du système informatique, Kalicommunication possède suffisamment d'applications pour satisfaire le client. Je n'ai donc pas eu a créer d'applications de toutes pièces. Les évolutions qui m'étaient attribuées concernaient surtout le site "Rapid-Flyer" et le PGI, utilisé par la PAO. Cependant ma mission était, avant tout, de garantir, avec le reste du service, le fonctionnement de l'architecture informatique. Pour cela j'ai eu l'occasion de développer sur la quasi-totalité des applications du système.

\section{L'administration système}
Certains problèmes n'étaient pas du ressort des développeurs. C'est l'administrateur système de Kalicommunication qui était en charge d'agir sur les serveurs et de résoudre les problèmes réseaux. En tant que développeurs nous devions lui transmettre ces problèmes. En revanche, dans certains cas nous ne connaissions pas la source du problème. Avant de transmettre le problème à l'administrateur système, nous devions enquêter sur l'origine du problème. Ayant accès aux machines hébergeant chaque application, nous pouvions effectuer les vérifications nécessaires avant de transmettre le problème à l'administrateur système.\newline
Faire ces vérifications demandait des connaissances en administration système et faisaient parties intégrantes de mes missions.