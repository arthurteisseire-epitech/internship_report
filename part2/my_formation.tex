\chap{Ma formation}

\section{Wordpress}
A mon arrivé Techniphoto avait besoin d'un site de présentation pour le groupe. La maquette du site n'étant pas encore prête à ce moment là, ma première mission était de m'entraîner à utiliser le système de gestion de contenu (CMS) Wordpress. Cette mission a été suspendu en raison du manque de développeur chez DataXplorer, comme évoqué dans la section précédente. Fin Août, début Septembre j'ai rejoins les développeurs au sein même de DataXplorer.

\section{DataXplorer}
DataXplorer est une startup qui loue des services informatiques pour divers entreprises. A mon arrivé elle était située à "La Plaine Images", puis elle a déménagée à "Euratechnologies". Ma mission chez DataXplorer était principalement de récupérer les connaissances du système informatique de Kalicommunication. Les développeurs de DataXplorer ont donc pris le temps, chacun leur tour, de m'expliquer le fonctionnement du système. Étant dans le secteur de l'imprimerie, ils m'ont enseignés les bases du secteur pour que je puisse comprendre le jargon employé au sein de l'entreprise. Cette phase d'apprentissage à duré environ un mois et demi. Évidemment j'ai continué d'apprendre tout au long de mon stage mais après cette "formation" j'étais autonome dans les tâches qui m'étaient attribuées.\newline
Après avoir compris de manière théorique le fonctionnement du système, mon apprentissage s'est fait par la pratique.
\section{Le support}
Le support faisant partie intégrante du service informatique chez Kalicommunication, j'ai pu concrétiser ce que les développeurs de DataXplorer m'avaient transmis. De manière général, le support constitue tous les problèmes urgents que rencontrent les clients ou les différents services de Kalicommunication. Les tâches de support demandent très peu de développements, voir dans certains cas, aucun.\newline
Pour pouvoir répondre aux demandes dans les plus brefs délais, nous avons utilisé un logiciel de gestion de tickets. Les tickets représentent les demandes des services de Kalicommunication, que se soit pour une correction ou pour une évolution. Ils permettent, entre autres, de se substituer des mails et de pouvoir garder une trace des problèmes, résolus ou non.\newline

\chap{Chapitre 2}
\section{Le développement}
\section{Kalicommunication}