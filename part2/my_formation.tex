\chap{Ma formation}

\section{Wordpress}
Lorsque j'ai intégré l'entreprise (début août), Techniphoto avait besoin d'un site de présentation pour le groupe. La maquette du site n'étant pas encore prête à ce moment-là, ma première mission était de m'entraîner à utiliser le système de gestion de contenu (CMS) Wordpress. Un CMS permet de créer un site web à partir d'une base toute faite. Faire un site Wordpress est donc très rapide mais est limité dans la personnalisation.\newline
Cette mission a été suspendue en raison du manque de développeur chez DataXplorer, comme évoqué dans la section précédente. Fin Août, début Septembre j'ai rejoint les développeurs au sein même de DataXplorer.

\section{DataXplorer}
DataXplorer est une startup qui loue des services informatiques pour diverses entreprises. A mon arrivé elle était située à "La Plaine Images", puis elle a déménagé à "Euratechnologies". Ma mission chez DataXplorer était principalement de récupérer les connaissances du système informatique de Kalicommunication. Les développeurs de DataXplorer ont donc pris le temps, chacun leur tour, de m'expliquer le fonctionnement du système. Étant dans le secteur de l'imprimerie, ils m'ont enseigné les bases du secteur afin que je puisse comprendre le jargon employé au sein de l'entreprise. Cette phase d'apprentissage a duré environ un mois et demi. Évidemment j'ai continué d'apprendre tout au long de mon stage. Suite à cette "formation" je suis devenu autonome dans les tâches qui m'étaient attribuées.\newline
Après avoir compris de manière théorique le fonctionnement du système, mon apprentissage s'est fait de façon empirique.
