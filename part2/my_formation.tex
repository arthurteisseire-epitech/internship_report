\chap{Ma formation}

\section{Wordpress}
A mon arrivé Techniphoto avait besoin d'un site "vitrine", uniquement pour présenter le groupe. En effet Techniphoto n'avait pas de site à proprement parler pour présenter ses différentes entreprises. La maquette du site n'étant pas encore prête à ce moment là, ma première mission était de m'entraîner à utiliser le système de gestion de contenu (CMS) Wordpress. Cette mission a été suspendu en raison du manque de développeur chez DataXplorer que nous avons évoqué dans la section précédente. Fin Août, début Septembre j'ai rejoins les développeurs au sein même de DataXplorer.

\section{DataXplorer}
DataXplorer est une startup qui loue des services informatiques pour divers entreprises. A mon arrivé DataXplorer était située à "La Plaine Images", puis elle a déménagée à Euratechnologies. Ma mission chez DataXplorer était principalement de récupérer les connaissances du système informatique de Kalicommunication. Les développeurs de DataXplorer ont donc pris le temps, chacun leur tour, de m'expliquer le fonctionnement du système. Étant dans le secteur de l'imprimerie, ils m'ont enseignés les bases du secteur pour que je puisse comprendre le jargon employé. Cette phase d'apprentissage à duré environ un mois et demi. Évidemment j'ai continué d'apprendre tout au long de mon stage mais après cette "formation" j'étais autonome dans les tâches qui m'étaient attribuées.\newline
Après avoir compris de manière théorique le fonctionnement du système, mon apprentissage c'est fait principalement en deux parties:
\begin{itemize}
\item Le support
\item Le développement
\end{itemize}

\section{Le support}
Le support faisant partie intégrante du service informatique chez Kalicommunication, c'est la première formation pratique que j'ai effectué. De manière général, le support constitue tous les problèmes urgents que rencontrent les clients ou les différents services de Kalicommunication.

\chap{Chapitre 2}
\section{Le développement}
\section{Kalicommunication}