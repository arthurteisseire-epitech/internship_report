\chap{Introduction}
Aujourd'hui, la tendance porterait à croire que l'imprimerie serait en voit de disparition face l'impact d'Internet et des nouvelles technologies. Il n'en fût rien. Bien au contraire. L'impression continue de s'adapter aux changements et de tirer le meilleur parti de ces évolutions.\newline

Du 10 août 2018 au 21 décembre 2018, j'ai effectué un stage au sein de l'entreprise Kalicommunication.
Située à Marquette-Lez-Lille, cette entreprise est spécialisée dans le secteur de l'impression en ligne.\newline
Ma mission était principalement de continuer le développement de leur site internet, "Rapid-Flyer".
Accompagné par Christophe Lenoir, directeur des services informatiques, j'ai ainsi pu développer mes compétences informatiques et comprendre le fonctionnement d'un commerce en ligne. Plus précisément, j'ai pu approfondir mes connaissances sur le fonctionnement du web. Au sein d'une équipe complète, j'ai également eu l'occasion de monter en compétence en administration système. Comprendre le fonctionnement de ces aspects était ma motivation première à rejoindre cette équipe.\newline
Au delà d'enrichir mes connaissances, ce stage a été une réelle opportunité car j'ai pu y découvrir la vie en entreprise qui m'était encore inconnue. J'ai également pu observer les différents services que nécessitait une entreprise. Faire preuve de professionnalisme et de rigueur m'est devenu indispensable pour travailler efficacement avec ces différents services.\newline

Dans un premier temps nous nous décririons l'entreprise et son secteur d'activité. Puis nous étudierons mes missions, pendant ce stage, avant d'en conclure avec celui-ci.