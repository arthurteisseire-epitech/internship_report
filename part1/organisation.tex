\chap{L'organisation}

\section{Le groupe}
A l'origine, Kalicommunication était une imprimerie indépendante et possédait sont propre site internet "Rapid-Flyer".\newline
En 201., Techniphoto, un groupe d'imprimeurs, a racheté Kalicommunication. Aujourd'hui, Kalicommunication fait toujours parti du groupe et ne possède plus d'imprimantes. Elle fait donc appel aux imprimeries du groupe pour imprimer les produits commandés sur le site Rapid-Flyer mais dispose de ses propres services pour le reste du processus de commande.

\section{Les services}
Kalicommunication est divisée en quatre pôles principaux: la Publication Assisté par Ordinateur (PAO), le marketing, le service client et le service informatique.\newline
La PAO est un service propre à l'imprimerie qui permet d'assurer le bon processus d'impression des documents. En effet, celui-ci détermine si le document fourni par le client est prêt ou non pour l'impression et procède aux retouches nécessaires. Une fois cette étape terminée, la PAO s'occupe du cheminement du produit allant de l'imprimeur jusqu'à la livraison. Ce service constitue donc un rôle indispensable pour des spécialistes de l'impression en ligne.\newline
De son côté, le service marketing met en place des stratégies pour attirer le client et améliore la visibilité du site.
Le service client, quant à lui, s'assure de la satisfaction du client en traitant leurs demandes.\newline
Enfin, le service informatique de Kalicommunication assure le bon fonctionnement du site Rapid-Flyer, du progiciel utilisé par la PAO, ainsi que de tous les outils nécessaires au bon cheminement du produit.

\section{L'architecture}
L'entièreté des revenus de Kalicommunication reposent sur son site "Rapid-Flyer". Cependant "Rapid-Flyer" ne représente qu'une brique du système informatique. C'est l'architecture de ce système qui assure le bon cheminement des produits. Il est donc important d'en expliquer le fonctionnement.

\newpage
\begin{figure}[p]
 \centering
\begin{center}
	\scalebox{1.7} {
		\begin{tikzpicture}[auto, node distance=1.5cm]
		    \node[entity] (rf) {Rapid-Flyer};
		    \node[entity] (pao) [below = of rf] {PAO};
		    \node[entity] (sst) [below = of pao] {Imprimeur};
		    \node[entity] (expe) [below = of sst] {Expédition};

			\draw[->, to path={-> (\tikztotarget)}]
		    	(rf) edge (pao) (pao) edge (sst) (sst) edge (expe);
		\end{tikzpicture}
	}
\vspace{1cm}
\end{center}
\end{figure}
\newpage

Ce schéma a été nettement simplifié pour comprendre rapidement le cheminement d'un produit commandé sur Rapid-Flyer. Comme nous l'avons vu plus haut, la PAO s'occupe de transmettre le produit à l'imprimeur. C'est ensuite la PAO qui s'occupe de livrer le produit chez le client. Nous pouvons compléter ce schéma en y ajoutant des éléments comme le schéma ci-dessous.

\begin{figure}[htb]
\begin{center}
	\scalebox{1.7} {
		\begin{tikzpicture}[auto, node distance=1.5cm]
		    \node[entity] (rf) {Rapid-Flyer};
		    \node[entity] (etl) [below = of rf] {ETL};
			\node[entity] (rprice) [right = of etl] {RPrice};
		    \node[entity] (erp) [below = of etl] {ERP};
		    \node[entity] (twist) [right = of erp] {Twist};
		    \node[entity] (metrix) [left = of erp] {Metrix};
		    \node[entity] (sst) [below = of erp] {Imprimeur};
		    \node[entity] (expe) [below = of sst] {Expédition};
	
			\draw[->, to path={-> (\tikztotarget)}]
		    	(rf) edge (etl) (etl) edge (erp) (erp) edge (sst) (sst) edge (expe)
				(etl) edge (rprice) (rprice) edge (etl)
				(erp) edge (twist) (twist) edge (erp)
				(erp) edge (metrix) (metrix) edge (erp);
		\end{tikzpicture}
	}
\vspace{1cm}
{\caption*{Schéma de l'architecture informatique de Kalicommunication}}
\end{center}
\end{figure}
\newpage

Sur ce nouveau schéma ont été rajoutées les principales applications développées par le service informatique de Kalicommunication. Ces nouvelles applications nécessitent des explications:
\begin{itemize}
\item RETL est une application qui permet de faire le pont entre les autres applications. Elle récupère des informations d'une application pour les envoyer dans une autre avec le bon format de données. Les autres applications n'ont donc pas à gérer ce processus.
\item L'application RPrice, quant à elle, calcule tous les prix nécessaires aux différents services. Le prix du produit, le coût de l'imprimeur, du transporteur etc. Elle occupe donc un rôle indispensable au sein du système.
\item La PAO, pour communiquer avec les autres applications, utilise un progiciel de gestion intégré (PGI). Ce logiciel permet à la PAO d'interagir avec les autres applications. Le PGI constitue donc le cœur du système informatique.
\item Twist est un des rares logiciels qui n'a pas été développé par Kalicommunication. Son rôle est d'automatiser la vérification des fichiers clients. Un gain de temps pour la PAO qui peut donc s'occuper du reste du flux.\newline
\end{itemize}
Nous connaissons ensuite la fin du processus. Il est bon de préciser que ce schéma représente une simplification de l'architecture informatique de Kalicommunication. Cependant une explication complète de cette architecture serait trop technique pour rentrer dans le cadre de ce rapport.