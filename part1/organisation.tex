\chap{L'organisation}

\section{Le groupe}
A l'origine, Kalicommunication était une imprimerie indépendante et possédait sont propre site internet "Rapid-Flyer".\newline
En 201., Techniphoto, un groupe d'imprimeurs, a racheté Kalicommunication. Aujourd'hui, Kalicommunication fait toujours parti du groupe et ne possède plus d'imprimantes. Elle fait donc appel aux imprimeries du groupe pour imprimer les produits commandés sur le site Rapid-Flyer mais dispose de ses propres services pour le reste du processus de commande.

\section{Les services}
Kalicommunication est divisée en quatre pôles principaux: la Publication Assisté par Ordinateur (PAO), le marketing, le service client et le service informatique.\newline
La PAO est un service propre à l'imprimerie qui permet d'assurer le bon processus d'impression des documents. En effet, celui-ci détermine si le document fourni par le client est prêt ou non pour l'impression et procède aux retouches nécessaires. Une fois cette étape terminée, la PAO s'occupe du cheminement du produit allant de l'imprimeur jusqu'à la livraison. Ce service constitue donc un rôle indispensable pour des spécialistes de l'impression en ligne.\newline
De son côté, le service marketing met en place des stratégies pour attirer le client et améliore la visibilité du site.
Le service client, quant à lui, s'assure de la satisfaction du client en traitant leur demandes.\newline
Enfin, le service informatique de Kalicommunication assure le bon fonctionnement du site Rapid-Flyer, du progiciel utilisé par la PAO, ainsi que de tous les outils nécessaires au bon cheminement du produit.