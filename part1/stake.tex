\chap{Les enjeux du groupe}

\section{Les imprimeries}
Chez les imprimeurs, acquérir des connaissances en informatique n'est souvent pas une priorité. La plupart des imprimeries du groupe manquent de connaissances et de matériels dans le domaine. Ce manque de connaissances a entraîné de mauvaises décisions faisant perdre des sommes importantes d'argent au groupe.\newline
A titre d'exemple, une imprimerie du groupe a investi une grande somme d'argent pour créer son site internet. Une fois le site internet terminé, ils se sont rendu compte que la version utilisée n'était déjà plus maintenu, avant même que le développement commence. Le site est pour l'instant fonctionnel mais il devra sans doute être entièrement re-développé prochainement.\newline
Pour éviter à nouveau ces pertes, le groupe à besoin d'un service informatique sur lequel toutes les imprimeries du groupe peuvent compter. La plupart des imprimeries dépensant leur argent pour leurs propres besoins, unifier ces dépenses s'impose. Christophe Lenoir tente donc d'étendre le service informatique de Kalicommunication pour que les imprimeries du groupe puissent se reposer sur celui-ci. De plus, ne pas dépendre de services externes permet d'avoir le contrôle de la situation en cas d'urgence. Le service informatique de Kalicommunication se doit donc d'être compétent dans de nombreux domaines.

\section{Le service informatique}
En <date>, Techniphoto a externalisé le service informatique de Kalicommunication chez Dataxplorer, une startup informatique. Ceci avait pour but de l'ajouter au sein du groupe. Cependant ce changement a fait naître un manque de communication entre le service informatique et les autres services de Kalicommunication.\newline
De plus, depuis le <date>, il n'y avait plus de directeur des services informatiques (DSI) chez Kalicommunication. C'était le directeur marketing qui était en charge des développeurs ainsi que de leur recrutement. Les développeurs n'avaient donc pas de supérieur hiérarchique à proprement parler et les priorités de développements n'étaient pas clairement définies. L'accumulation de ces problèmes entraînait un ralentissement des développements informatiques.\newline
En Juillet 2018, Christophe Lenoir a pris en charge le service informatique de Kalicommunication. Il a démarré le processus de recrutement de développeurs dans le but de ré-internaliser le service informatique.\newline
Cependant, en septembre 2018, deux des trois développeurs de ce même service informatique ont posé leur démission. Leur période de préavis était seulement d'un mois. A ce moment Dataxplorer n'avait pas d'autre développeur à proposer et j'étais le seul développeur en interne chez Kalicommunication. Christophe Lenoir, n'ayant pas prévu cette situation, a donc décidé de m'envoyer chez Dataxplorer pendant un mois afin que les développeurs me transmettent leurs connaissances.\newline
Ma mission principale était donc de récupérer un maximum de connaissances sur l'organisation du système informatique de Kalicommunication afin de pouvoir les transmettre, à mon tour, aux nouveaux arrivants.