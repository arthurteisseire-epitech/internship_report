\documentclass[a4paper]{report}
\usepackage[T1]{fontenc}
\usepackage[frenchb]{babel}
\usepackage[utf8]{inputenc}
\usepackage{graphicx}
\usepackage[top=3cm, bottom=3cm, left=3cm, right=3cm]{geometry}

\title{Rapport De stage}
\author{Arthur Teisseire}
\begin{document}
\renewcommand{\contentsname}{Sommaire}
\renewcommand{\thesection}{\arabic{section}}
\tableofcontents
\chapter*{Introduction}
\addcontentsline{toc}{chapter}{Introduction} \markboth{INTRODUCTION}{}
Du 10 août 2018 au 21 décembre 2018, j'ai effectué un stage au sein de l'entreprise Kalicommunication.
Située à Marquette-Lez-Lille, cette entreprise est spécialisée dans le secteur de l'impression en ligne.\newline
Ma mission était principalement de continuer le développement de leur site internet, "Rapid-Flyer".
Accompagné par Christophe Lenoir, directeur des services informatiques, j'ai ainsi pu développer mes compétences informatiques en entreprise et découvrir le fonctionnement d'un commerce en ligne.
\part{L'entreprise}
\chapter*{L'organisation}
\addcontentsline{toc}{chapter}{L'organisation} \markboth{L'organisation}{}
\section{Le groupe}
A l'origine, Kalicommunication était une imprimerie indépendante et possédait sont propre site internet "Rapid-Flyer".\newline
En 201., Techniphoto, un groupe d'imprimeurs, a racheté Kalicommunication. Aujourd'hui, Kalicommunication fait toujours parti du groupe et ne possède plus d'imprimantes. Elle fait donc appel aux imprimeries du groupe pour imprimer les produits commandés sur le site Rapid-Flyer mais dispose de ses propres services pour le reste du processus de commande.
\section{Les services}
Kalicommunication est divisé en quatre pôles principaux: la Publication Assisté par Ordinateur (PAO), le marketing, le service client et le service informatique.\newline
La PAO est un service propre à l'imprimerie permettant d'assurer le bon processus d'impression des documents. Celui-ci détermine si le document fourni par le client est prêt pour l'impression ou non et y fait les retouches nécessaires. Une fois cette étape terminée, la PAO s'occupe du cheminement du produit allant de l'imprimeur jusqu'à la livraison. Ce service constitue donc un rôle indispensable pour des spécialistes de l'impression en ligne.\newline
De son côté, le service marketing met en place des stratégies pour attirer le client et apporter une meilleure visibilité au site.\newline
Le service client va, quant à lui, s'assurer de la satisfaction du client en répondant à ses demandes.\newline
Le service informatique de Kalicommunication s'occupe du site Rapid-Flyer, du progiciel utilisé par la PAO, ainsi que de tous les outils nécessaires au bon cheminement du produit.
\chapter*{Problématiques de l'entreprise}
\addcontentsline{toc}{chapter}{Problématiques de l'entreprise} \markboth{Problématiques de l'entreprise}{}
\setcounter{section}{0}

\section{Le service informatique}
Dans une optique de séparation des services, Techniphoto a décidé de d'externaliser le service informatique de Kalicommunication dans une entreprise. La communication est donc devenu plus compliqué entre le service informatique et les autres services de Kalicommunication. Ce problème de communication a donc ralenti les développements.

Christophe Lenoir a donc décidé de récupérer le service informatique au sein de Kalicommunication. Pour cela il a commencé le recrutement des développeurs afin de ne plus être dépendant de l'entreprise qui lui fournissait ce service. Cependant recruter des développeurs dans le secteur de l'imprimerie peut prendre du temps. Il a donc fait le choix de recruter un stagiaire pendant cette phase de transition. Ma mission était donc de récupérer le maximum de connaissances du service informatique afin de pouvoir les transmettre à mon tour au nouveaux développeurs.

\part{Ma mission}
\end{document}