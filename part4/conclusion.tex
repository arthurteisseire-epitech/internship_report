Ce stage m'a été très enrichissant car il m'a permit de découvrir le monde de l'imprimerie, ses acteurs, ses secteurs d'activités et ses contraintes. J'ai pu y appliquer concrètement ce que j'ai appris à Epitech, ce que j'ai particulièrement apprécié. De plus, avant ce stage j'avais une mauvaise perception du développement web. J'y ai appris de très nombreuses choses intéressantes concernant le web. Développeur d'application web est rentré dans mes choix possible de carrière professionnel.\newline

L'entreprise Kalicommunication est entré dans une période de changement : arrivé de Christophe Lenoir, nouveau service informatique, nouveau directeur général. Je suis très fier d'avoir pu participer et contribuer à ce changement. L'évolution continue du service informatique de Kalicommunication pourra permettre à toutes les imprimeries du groupe Techniphoto de se réunir autour de celui-ci. Cette avantage permettrait de faire économies conséquentes pour le groupe Techniphoto.\newline

Fort de cette expérience j'aimerais m'orienter via le part-time de troisième année vers le développement logiciel. Web ou non. Une petite entreprise serait optimale car celle-ci m'a permit d'être en contact direct avec le client. Répondre de mon mieux aux besoins des clients a enrichi ma vision de la programmation et je souhaite continuer dans cette voie.