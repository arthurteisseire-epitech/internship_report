\chap{Mes corrections}

\section{Le support}
Un problème récurrent existait chez Kalicommunication : les fichiers des clients ne parvenait pas jusqu'à l'application qu'utilisait la PAO. Pour que la PAO puisse continuer son travail, nous devions récupérer les fichiers présent sur le site "Rapid-Flyer" et les téléverser sur l'application de la PAO.\newline
Rapatrier ces fichiers devenait un problème pour les développeurs car une partie de notre temps de développement y était consacré.\newline
On m'a donc confié la tâche de comprendre d'où venait ce problème. Malheureusement c'était un problème très complexe qui n'avait jamais réussi à être corrigé, même par des développeurs expérimentés. J'ai donc développé un système permettant à la PAO de rapatrier les fichiers facilement sans avoir besoin de faire appel à nous. Le problème de fond n'était pas corrigé mais ça a permit à la PAO et au service informatique de gagner du temps.

\section{Les devis}
Un devis est demandé par un client lorsque le produit recherché n'est pas présent sur le site "Rapid-Flyer". Le service client s'occupe alors de faire le devis via une interface dédié sur le site. Cependant certains devis posait problème et le service client se retrouvait dans l'incapacité de les créer.\newline
J'ai donc été chargé de corriger ces différents problèmes. La plus grande difficulté que j'ai rencontré sur ce problème était d'identifier sa source car la plupart des devis fonctionnaient. Le problème était ensuite facile à résoudre. A ma connaissance, il n'y a plus eu de problème sur les devis après cette correction.

\section{Les bons de commandes}
Pendant mon stage la PAO utilisait deux progiciels de gestions intégrés. L'ancien et le nouveau. L'ancien était encore utilisé car le nouveau n'était pas encore opérationnel à 100\%. Entre autres, les bons de commandes envoyés aux imprimeurs n'était pas fiables. Un développeur de DataXplorer a corrigé le problème, mais les anciens bons de commandes n'avaient pas les bonnes informations. Mon travail ici était d'écrire un programme permettant de modifier tous les bons de commandes existants. Cette mission était très délicate car mon programme devait modifier directement tous les bons de commandes en production. La moindre erreur pouvait poser des problèmes. J'ai donc pris le temps de tester mon programme à de nombreuses reprises avant de le lancer en production. La plus grande difficulté que j'ai rencontré était de tester mon programme sur un autre environnement que celui de production. Certains détails étaient différents entre deux environnements.

\section{Les quantités personnalisées}
Les produits sur "Rapid-Flyer" ont différents prix prédéfinis. Ces prix étant fixes, le travail de la PAO et du sous-traitants (ex: l'imprimeur) était simplifié.\newline
Le marketing a proposé d'ajouter des quantités personnalisées. A mon arrivé le développement était déjà en place mais il n'était pas fonctionnel. J'ai donc eu pour mission de corriger ce problème. La difficulté ici était de comprendre le fonctionnement des produits sur "Rapid-Flyer". Une fois ceci fait la correction n'était pas complexe.

\section{Les planches}
En imprimerie, une planche est un ensemble d'images (fichiers clients) disposées sur une même page. Cela permet, en une seule impression, d'obtenir plusieurs fichiers clients, et ainsi optimiser les coûts. Cette action est appelé l'amalgame. Elle peut être complexe car les différents clients commandent rarement la même quantité de produits. C'est la PAO qui effectue ces amalgames. Ceci étant complexe, il est fréquent de supprimer des planches et de les construire a nouveau.\newline
Le progiciel de gestion intégré qu'utilisait la PAO avait la fonctionnalité de supprimer des planches. Cependant ces planches n'étaient pas supprimées correctement et engendraient des problèmes ultérieurement. Corriger ceci m'a permit de comprendre des concepts de programmations qui m'étaient encore inconnus.