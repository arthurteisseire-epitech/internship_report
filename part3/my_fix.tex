\chap{Mes corrections}

\section{Le site}
La partie visible du site est très importante. En effet c'est la première étape que visualise le client. La plupart des problèmes liés à la visualisation du site nous sont transmis par le service marketing. Ces difficultés étant majoritairement mineures et peu fréquentes, leurs corrections m'étaient attribuées au début de mon stage. Elles pouvaient être en rapport avec le référencement ou encore avec le visuel de certains éléments.\newline
A titre d'exemple, la barre de navigation du site était trop petite pour intégrer une nouvelle catégorie de produits. Plusieurs possibilités se présentaient : je pouvais, par exemple¸ réduire la police d'écriture au sein de la barre ou encore élargir cette dernière. Pour faire le choix le plus adapté, j'ai fait appel au service marketing en exposant les différentes solutions. Après discussion avec cette équipe, j'ai développé la solution choisie.\newline
Cela m'a permis de comprendre que de déterminer précisément le besoin du client est une priorité avant de commencer un développement.

\section{Le support}
Un problème récurrent existait chez Kalicommunication : les fichiers des clients ne parvenaient pas jusqu'à l'application qu'utilisait la PAO. Pour qu'elle puisse mener à bien son travail, nous devions récupérer les fichiers présents sur le site "Rapid-Flyer" et les téléverser sur l'application de la PAO.\newline
Rapatrier ces fichiers ralentissait notre travail car une partie de notre temps devait y être consacrée.\newline
Une de mes tâches était de comprendre d'où venait cette anomalie. J'ai développé un système permettant à la PAO de rapatrier les fichiers facilement, sans avoir besoin de faire appel aux développeurs. Le problème de fond n'était pas corrigé, mais cela nous a permis de continuer à en chercher l'origine, sans interférence avec la PAO.\newline
Nous avons fini par découvrir et corriger la source du problème peu avant la fin de mon stage.

\section{Les devis}
Un devis est demandé par un client lorsque le produit recherché n'est pas présent sur le site "Rapid-Flyer". Le service client s'occupe alors de le faire via une interface dédié sur le site. Cependant, dans certains cas, le service client se retrouvait dans l'incapacité de les créer.\newline
J'ai donc été chargé de corriger ce problème. La plus grande difficulté que j'ai rencontrée était d'identifier sa source, car la plupart des devis fonctionnaient. Ceci étant fait, le problème était ensuite facile à résoudre. A ma connaissance, tous les devis fonctionnaient correctement après correction.

\section{Les bons de commandes}
Pendant mon stage, la PAO utilisait deux progiciels de gestions intégrés. L'ancien et le nouveau. L'ancien était toujours utilisé car le nouveau n'était pas encore opérationnel à 100\%. Entre autres, les bons de commandes envoyés aux imprimeurs n'étaient pas fiables. Un développeur de DataXplorer a corrigé le problème, mais les anciens bons de commandes n'avaient pas les bonnes informations. Mon travail ici était d'écrire un programme permettant de modifier tous les bons de commandes existant. Cette mission était très délicate car mon programme devait modifier directement tous les bons de commandes en production. Une erreur aurait pu freiner cette dernière. J'ai donc pris le temps de tester mon programme à de nombreuses reprises avant de le lancer en production. La principale complexité était d'expérimenter les conséquences de mon programme sur un autre environnement que celui de la production. En effet, certains détails étaient différents entre ces deux environnements.

\section{Les quantités personnalisées}
Les produits sur "Rapid-Flyer" ont différents prix prédéfinis. Ces prix étant fixes, le travail de la PAO et du sous-traitants (ex: l'imprimeur) était simplifié.\newline
Le marketing a proposé d'ajouter des quantités personnalisées. A mon arrivée, le développement était déjà en place mais n'était pas fonctionnel. J'ai donc eu pour mission de le rendre opérationnel. La difficulté ici était de comprendre le fonctionnement des options des produits sur "Rapid-Flyer". Ceci étant fait, la correction fut triviale.

\section{Les planches}
En imprimerie, une planche est un ensemble d'images -- fichiers clients -- disposé sur une même page. Cela permet, en une seule impression, d'obtenir plusieurs fichiers clients, et ainsi d'optimiser les coûts. Cette action est appelée "amalgame". Elle peut être complexe car les différents clients commandent rarement la même quantité de produits. C'est la PAO qui effectue ces amalgames. Au vu de leur complexité, il est fréquent de supprimer des planches et de les construire à nouveau.\newline
Le progiciel de gestion intégré qu'utilisait la PAO avait la fonctionnalité de supprimer des planches. Cependant, ces dernières n'étaient pas supprimées correctement et engendraient des problèmes ultérieurement. Corriger ceci m'a permis de comprendre des concepts de programmations qui m'étaient encore inconnus.