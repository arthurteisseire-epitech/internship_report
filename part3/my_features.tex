\chap{Mes évolutions}

\section{Wordpress}
Comme évoqué dans le chapitre précédent, ma première mission au sein de Kalicommunication était de créer un site de présentation pour Techniphoto. La demande était de faire un site sur une seule page. Wordpress m'était encore inconnu et je découvrais le fonctionnement d'un système de gestion de contenu. Pendant la première semaine, j'ai consacré mon temps à comprendre ce fonctionnement. Pour me former, je commençais à créer un site pour Techniphoto. La plus grande difficulté  était de construire l'apparence du site sans maquettes. Elles n'étaient pas encore construites.\newline
Après une semaine de travail, le site était terminé. Il m'a été demandé d'en faire un deuxième avec un thème différent. Ceci avait pour objectif, non seulement, de m'entraîner à nouveau, mais aussi que le demandeur puisse orienter son choix.\newline
Les sites étaient terminés mais je n'ai pas eu le temps de les présenter au demandeur car je devais me rendre chez DataXplorer.
Peu avant la fin de mon stage, une maquette a été construite et j'ai donc pu réaliser le site avec les images de celle-ci. Le site n'est pas encore en ligne. J'ai transmis ma réalisation aux nouveaux développeurs de Kalicommunication.

\section{Migration}
Pour garder une trace de nos développements, nous utilisions un système de contrôle de version. A mon arrivée, nos développements étaient stockés sur un serveur. Pour des raisons économiques, Christophe Lenoir a décidé de changer de serveur. Une fois que le nouveau serveur a été configuré par les administrateurs systèmes, on m'a confié la tâche de migrer toutes les applications sur ce serveur. Aujourd'hui, l'ancien serveur n'est plus utilisé et a été entièrement remplacé par le nouveau.

\section{Les mises en production}
Au sein du service informatique chez Kalicommunication, les mises en production ont lieu une fois par semaine. La mise en production consiste à appliquer les modifications et les ajouts des développeurs à l'application utilisée par le client final. Plus simplement, avant celle-ci, les changements que font les développeurs restent uniquement sur leurs ordinateurs personnels. L'intérêt de faire une mise en production par semaine est de vérifier que les modifications effectuées ne provoquent pas de problèmes, permettant de revenir en arrière le cas échéant.\newline
Après avoir appris a faire ces mises en productions, elles m'étaient assignées. Pour savoir quels tickets étaient prêts pour la mise en production, nous avions besoin que le demandeur valide le fonctionnement du changement effectué. De notre côté -- les développeurs -- nous devions fournir les informations nécessaires dans les tickets pour que celui chargé de la mise en production puisse récupérer les changements de chacun. Une fois ces changements récupérés localement, nous pouvions les transférer à l'application en production. La dernière étape était d'informer, dans le ticket, que le changement a bien été mis en production.

\section{Transmission de connaissances}
Depuis mon arrivée, deux développeurs ont rejoint le service de Kalicommunication. Avant qu'ils nous rejoignent, il restait un développeur de DataXplorer et moi-même. À chaque arrivée, mon rôle était de leur transmettre le maximum de connaissances que j'avais pu acquérir à l'aide des développeurs de DataXplorer et des différentes missions que l'on m'avait attribuées. L'architecture de Kalicommunication étant complexe, je n'ai pas eu le temps de travailler sur toutes ces applications. C'est le développeur de DataXplorer qui se chargeait de transmettre les connaissances qui me manquaient.

\section{Les statistiques marketing}
Pour cette mission j'étais en relation avec le service marketing. Ce dernier avait fait appel à une entreprise extérieure pour récupérer les statistiques du site via l'interface Google Analytics. Celle-ci permet de visualiser des statistiques.\newline
J'ai travaillé avec un prestataire de cette entreprise pour mener à bien cette mission. Il s'est occupé de la partie administrative de Google Analytics. De mon côté, j'ai développé un module pour envoyer les informations demandées par le marketing à Google. Ce développement a pris du temps car je n'avais aucune connaissance dans ce domaine. De surcroît, le prestataire est parti en congé deux semaines et des évolutions ont été ajoutées au fur et à mesure. Aujourd'hui ce module est en production sur le site "Rapid-Flyer" et le marketing peut profiter de ces statistiques via l'interface Google Analytics.

\section{Les factures}
Pour ma dernière mission, j'ai travaillé en collaboration avec Damien Merchier, un des développeurs web du nouveau service informatique de Kalicommunication.\newline
Le gouvernement ayant ajouté des lois sur les factures électroniques, nous devions changer leur mode de fonctionnement. Pour respecter les nouvelles normes imposées, les factures des clients de l'entreprise ont dû être stockées dans une entreprise extérieure. Ceci permettant de garantir que l'entreprise (ici Kalicommunication) ne puisse pas modifier ses propres factures.\newline
Damien a donc créé une nouvelle application permettant d'envoyer les factures de Kalicommunication à l'entreprise externe. Pour que cette application fonctionne, des changements étaient nécessaires sur le site "Rapid-Flyer". J'ai donc aidé Damien à la création de la nouvelle application et effectué les modifications nécessaires sur le site. Celles-ci ne sont pas encore en production. Elles devraient l'être à partir de début janvier.