\documentclass[a4paper]{report}
\usepackage[T1]{fontenc}
\usepackage[frenchb]{babel}
\usepackage[utf8]{inputenc}
\usepackage[top=3cm, bottom=3cm, left=3cm, right=3cm]{geometry}
\begin{document}
\part{Le contexte}
\chapter{Présentation}
\section{L'organisation}
RapidFlyer est un site d’imprimerie en ligne accessible à l’adresse www.rapid-flyer.com. Ce site appartient à l’entreprise Kalicommunication. Kalicommunation appartient au groupe Techniphoto. 
Techniphoto est le 1er groupe d'imprimeur du Nord. 
Le groupe est créé en 1971 puis repris par Vincent Dufour. Les différentes imprimeries du goupe sont : IJB (Imprimerie Jean Bernard), ID (Impression Directe), Numériprint, La Monsoise, Ciscom, DataXplorer, Sopédi, Technofa, FG (Fidélité Graphique), Nord Imprim. 

\section{Le site}
Le site RapidFlyer propose à ces clients différentes sorte de produits tels que des affiches, des cartes de visites, des drapaux publicitaires, des flyers et d’autres. 
Afin de communiquer au mieux avec ses sous-traitants (imprimeurs), Kalicommunication dipose d’un ERP (progiciel de gestion intégré), RapidPM, afin de gérer toutes les commandes, les fichiers des clients non conformes, l’envoie aux sous-traitants et de nombreuses autres nécessitées. Ce service est appelé la PAO. RapidPM tend à être remplacer par un nouvel ERP, Rprint, plus récent et surtout plus configurable pour les membres de la PAO. Rprint permet donc aux membres de la PAO de ne plus être dépendant des développeurs à la moindre configuration. A mon arrivé le passage des produits d’RPM à Rprint était en cours. Cependant des bugs sur Rprint ont retardé le transfert. % Une fois le stage terminé ne pas oublier de mentionner l’etat des ERP. 

\chapter{L'organisation}

\end{document}
