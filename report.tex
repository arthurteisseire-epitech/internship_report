\documentclass[a4paper]{report}
\usepackage[T1]{fontenc}
\usepackage[frenchb]{babel}
\usepackage[utf8]{inputenc}
\usepackage[top=3cm, bottom=3cm, left=3cm, right=3cm]{geometry}

\title{Rapport De stage}
\author{Arthur Teisseire}
\begin{document}
\tableofcontents
\part{L'entreprise}
\chapter{Présentation}
\section{Introduction}
Du 10 août 2018 au 21 décembre 2018, j'ai effectué un stage au sein de l'entreprise Kalicommunication, plus précisement pour leur site d'impression en ligne Rapid-Flyer. Au cours de ce stage j'ai pu découvrir le fonctionnement d'un commerce en ligne.\newline
L'entreprise Kalicommunication se situe à Marquette-Lez-Lille. Elle est spécialisée dans le secteur de l'impression en ligne.\newline
Accompagné par Christophe Lenoir, directeur des services informatiques j'ai pu développer mes compétences informatique en entreprise. Ma mission  était principalement de continuer le developpement du site Rapid-Flyer.
\section{L'organisation}
Techniphoto est le premier groupe d'imprimeur du Nord. Le groupe est constitué de neuf %à vérifier
imprimeries et de Kalicommunication. Rapid-Flyer est le site marchand principal de ces imprimeries. Les clients commandent sur le site Rapid-Flyer et le service de Kalicommunication s'occupe de tout le cheminement des produits jusqu'à la livraison chez le client. Cependant Kalicommunication ne possède pas d'imprimantes et fait donc appel aux sous-traitants du groupe pour imprimer les produits des clients.

\section{Le site}
Le site RapidFlyer propose à ces clients différentes sorte de produits tels que des affiches, des cartes de visites, des drapaux publicitaires, des flyers et d’autres. 
Afin de communiquer au mieux avec ses sous-traitants (imprimeurs), Kalicommunication dipose d’un ERP (progiciel de gestion intégré), RapidPM, afin de gérer toutes les commandes, les fichiers des clients non conformes, l’envoie aux sous-traitants et de nombreuses autres nécessitées. Ce service est appelé la PAO. RapidPM tend à être remplacer par un nouvel ERP, Rprint, plus récent et surtout plus configurable pour les membres de la PAO. Rprint permet donc aux membres de la PAO de ne plus être dépendant des développeurs à la moindre configuration. A mon arrivé le passage des produits d’RPM à Rprint était en cours. Cependant des bugs sur Rprint ont retardé le transfert. % Une fois le stage terminé ne pas oublier de mentionner l’etat des ERP. 

\chapter{L'organisation}

\end{document}
