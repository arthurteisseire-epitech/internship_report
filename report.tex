\documentclass[a4paper]{report}
\usepackage[T1]{fontenc}
\usepackage[frenchb]{babel}
\usepackage[utf8]{inputenc}
\usepackage[top=3cm, bottom=3cm, left=3cm, right=3cm]{geometry}

\title{Rapport De stage}
\author{Arthur Teisseire}
\begin{document}
\tableofcontents
\part{L'entreprise}
\chapter{Présentation}
\section{Introduction}
Du 10 août 2018 au 21 décembre 2018, j'ai effectué un stage au sein de l'entreprise Kalicommunication, plus précisement pour leur site d'impression en ligne Rapid-Flyer. Au cours de ce stage j'ai pu découvrir le fonctionnement d'un commerce en ligne.\newline
L'entreprise Kalicommunication se situe à Marquette-Lez-Lille. Elle est spécialisée dans le secteur de l'impression en ligne.\newline
Accompagné par Christophe Lenoir, directeur des services informatiques, j'ai pu développer mes compétences informatique en entreprise. Ma mission était principalement de continuer le développement du site Rapid-Flyer.
\section{L'organisation}
Kalicommunication appartient au groupe Techniphoto. De nombreuses imprimeries font également parti du groupe, ce qui constitue un avantage de taille pour Kalicommunication. En effet, Kalicommunication fait appel aux imprimeries du groupe pour imprimer les produits commandé sur le site Rapid-Flyer.\newline
Kalicommunication est divisé en quatre pôles principaux: la PAO, le marketing, le service client et le service informatique.\newline
La PAO (publication assisté par ordinateur) est un service propre à l'imprimerie qui assure le bon processus d'impression des documents. C'est lui qui détermine si le document fourni par le client est prêt pour l'impression ou non et fait les retouches nécessaires si possible. Ensuite, la PAO s'occupe du cheminement du produit passant par l'imprimeur jusqu'à la livraison. Ce service constitue donc un rôle indispensable pour des spécialiste de l'impression en ligne.\newline
Le service marketing met en place des stratégies pour attirer le client et l'inciter à acheter les produits proposés par le service lui-même.\newline
Le service client va, quant à lui, s'assurer de la satisfaction du client en répondant à leur demandes, que se soit par mail ou par téléphone.\newline
Le service informatique de Kalicommunication s'occupe du site Rapid-Flyer, du progiciel utilisé par la PAO, ainsi que de tous les outils nécessaires au bon cheminement du produit.
\section{Le site}
Le site RapidFlyer propose à ces clients différentes sorte de produits tels que des affiches, des cartes de visites, des drapaux publicitaires, des flyers et d’autres. 
Afin de communiquer au mieux avec ses sous-traitants (imprimeurs), Kalicommunication dipose d’un ERP (progiciel de gestion intégré), RapidPM, afin de gérer toutes les commandes, les fichiers des clients non conformes, l’envoie aux sous-traitants et de nombreuses autres nécessitées. Ce service est appelé la PAO. RapidPM tend à être remplacer par un nouvel ERP, Rprint, plus récent et surtout plus configurable pour les membres de la PAO. Rprint permet donc aux membres de la PAO de ne plus être dépendant des développeurs à la moindre configuration. A mon arrivé le passage des produits d’RPM à Rprint était en cours. Cependant des bugs sur Rprint ont retardé le transfert. % Une fois le stage terminé ne pas oublier de mentionner l’etat des ERP. 

\chapter{L'organisation}

\end{document}